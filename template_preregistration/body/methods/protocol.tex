\subsection{Protocol}
\subsubsection{Study Design}
\begin{prereg}
\begin{instruction}
e.g. Field study or lab study, within or between-participant design, cross-over, etc.
Describe the type of study, e.g. if it is a clinical trial and if it belongs to a certain risk category.
Intervention-controlled design or observation study? If an intervention is administered, describe potential group allocation processes, potential cross-over, etc. 
Within- or between-participant or mixed study design? Describe if the study is a within or between (or a mix of a within and between) subject design. How will the participants be assigned to the conditions that are tested? 
Illustration recommended.
\end{instruction}
\end{prereg}

% Explain here why the proposed study design and tests are suitable to address the particular research question. What is the advantage and the limitation of the proposed protocol? 
\begin{table}[h]
\centering
\caption{Measurement schedule\label{tbl_schedule}}
%\begin{tabular}{@{}llll@{}}
\begin{tabularx}{\textwidth}{llllll}
\toprule
Measurement modality & Sampling frequency          & Timing of sampling & N \\ 
\midrule
Salivary melatonin   & 10 measurements per evening & Every 30 minutes   & 10                     \\
PVT                  & 10 measurements per evening & Every 30 minutes   & 10                     \\
KSS                  & 10 measurements per evening & Every 30 minutes   & 10                     \\ 
\bottomrule
\end{tabularx}
\end{table}


\subsubsection{Environment and Context}
%% [3.1.2 CIE TN 011:2020 / 4b Consort statement]
\begin{prereg}
\begin{instruction}
Define the physical context of the study. For laboratory and field interventions this can look very different. In the laboratory this should be done in greater detail concerning the spatial properties. If you use multiple rooms, these should be described separately to avoid confusion. We recommend that you include pictures and describe them in text. The environment could include: 
\begin{itemize}
    \item Season, location, time zone, multicenter vs. single location
    \item Settings (home, workplace, outdoors)
    \item Description of room (qualitative: windows, surface colours, quantitative: dimensions, orientation, air temperature)
    \item Lighting (position and size of the luminaires, windows and daylighting systems, position of the observer), optional: include table of the room measurements (e.g., alpha-opic irradiance values in reference to the participants eye position)
    \item Pictures/Illustration of the set-up
\end{itemize}
\end{instruction}
\end{prereg}


\subsubsection{Screening}
\begin{prereg}
\begin{instruction}
Describe the mode (online questionnaire, phone call, etc) and timing of the screening. By referring to Table \cref{tbl_incl_excl}, specify the instruments and cut-off used in a screening process according to inclusion/ exclusion criteria.
\end{instruction}
\end{prereg}

\subsubsection{Procedure}

\begin{prereg}
\begin{instruction}
Give an overview of the entire procedure, from the ambulatory part (if applicable) to the end of the study. E.g. numbers of conditions, different stages in each condition (ambulatory/ laboratory), numbers of each stage in each condition, the detailed timeline of all the intervention administration, tasks and measurements taking places from the arrival till the end of the study. A gantt chart showing the study schedule should be included.  
\end{instruction}
\end{prereg}


\subsubsection{Timeline}

\begin{prereg}
\begin{instruction}
Rough overview of all study appointments, start and end of the project. Refer to section “procedure” for detailed description. 
\end{instruction}
\end{prereg}


\subsubsection{Randomization}

\begin{prereg}
\begin{instruction}
Explain how the randomisation is conducted. Disclose if a non-random procedure is required (e.g. in order to balance back the matching samples). In addition, if the sample size does not allow a counterbalanced randomisation (e.g. $3$ conditions forming 6 types of sequences, but only $20$ participants are recruited), the samples of each sequence should be described.
\end{instruction}
\end{prereg}