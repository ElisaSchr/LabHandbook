\subsection{Outcomes}

\subsubsection{Primary Outcomes}
%% [CONSORT 6a]
\begin{prereg}
\begin{instruction}
Define all primary measurements in detail that are monitored according to research questions. The primary outcomes should be explicitly determined and also could be Linked to hypothesis or exploratory analysis .The informations of outcomes could include (\cref{tbl_outcomes}) :
\begin{itemize}
    \item Measurement modality
    \item Derived outcome measure and unit
    \item Number of measurements per participant
    \item Preprocessing
    \item Linked hypothesis or exploratory analysis
\end{itemize}
\end{instruction}
\end{prereg}

\subsection{Outcomes}

\subsubsection{Primary Outcomes}
%% [CONSORT 6a]
\begin{prereg}
\begin{instruction}
Define all primary measurements in detail that are monitored according to research questions. The primary outcomes should be explicitly determined and also could be Linked to hypothesis or exploratory analysis .The informations of outcomes could include (\cref{tbl_outcomes}) :
% * Measurement modality
% * Derived outcome measure and unit
% * Number of measurements per participant
% * Preprocessing
% * Linked hypothesis or exploratory analysis
\end{instruction}
\end{prereg}

\subsection{Outcomes}

\subsubsection{Primary Outcomes}
%% [CONSORT 6a]
\begin{prereg}
\begin{instruction}
Define all primary measurements in detail that are monitored according to research questions. The primary outcomes should be explicitly determined and also could be Linked to hypothesis or exploratory analysis .The informations of outcomes could include (\cref{tbl_outcomes}) :
% * Measurement modality
% * Derived outcome measure and unit
% * Number of measurements per participant
% * Preprocessing
% * Linked hypothesis or exploratory analysis
\end{instruction}
\end{prereg}

\subsection{Outcomes}

\subsubsection{Primary Outcomes}
%% [CONSORT 6a]
\begin{prereg}
\begin{instruction}
Define all primary measurements in detail that are monitored according to research questions. The primary outcomes should be explicitly determined and also could be Linked to hypothesis or exploratory analysis .The informations of outcomes could include (\cref{tbl_outcomes}) :
% * Measurement modality
% * Derived outcome measure and unit
% * Number of measurements per participant
% * Preprocessing
% * Linked hypothesis or exploratory analysis
\end{instruction}
\end{prereg}

\input{tbl/outcomes}


\subsubsection{Covariates}
%% Covariates [CONSORT 12b “adjusted analysis”]
\begin{prereg}
\begin{instruction}
Measurements which are not directly linked to the research question, but they are used in the analyses with primary factors to gain the precision of data interpretation. The addition of covariates should be justified by evidence. E.g. 
% * Relevant demographic features (e.g. age for child and adolescent population),
% * Order of condition (if the conditions are not counterbalanced)
\end{instruction}
\end{prereg}


\subsubsection{Other Measurements}
\begin{prereg}
\begin{instruction}
Measurements which are not directly linked to the research question, but they can provide information about the quality of the study. The instruments and the purpose of the measurements should be addressed. Such as:
% * Baseline demographic and clinical characteristics for each group [CONSORT 15]
% * Adverse events [CONSORT 19]
% * Expectancy (if blinded)
% * Measurements which monitor the compliance of the participants to restrictions (e.g. caffeine levels)
% * SOMETHING FOR LIGHT STUDIES?
\end{instruction}
\end{prereg}



\subsubsection{Covariates}
%% Covariates [CONSORT 12b “adjusted analysis”]
\begin{prereg}
\begin{instruction}
Measurements which are not directly linked to the research question, but they are used in the analyses with primary factors to gain the precision of data interpretation. The addition of covariates should be justified by evidence. E.g. 
% * Relevant demographic features (e.g. age for child and adolescent population),
% * Order of condition (if the conditions are not counterbalanced)
\end{instruction}
\end{prereg}


\subsubsection{Other Measurements}
\begin{prereg}
\begin{instruction}
Measurements which are not directly linked to the research question, but they can provide information about the quality of the study. The instruments and the purpose of the measurements should be addressed. Such as:
% * Baseline demographic and clinical characteristics for each group [CONSORT 15]
% * Adverse events [CONSORT 19]
% * Expectancy (if blinded)
% * Measurements which monitor the compliance of the participants to restrictions (e.g. caffeine levels)
% * SOMETHING FOR LIGHT STUDIES?
\end{instruction}
\end{prereg}



\subsubsection{Covariates}
%% Covariates [CONSORT 12b “adjusted analysis”]
\begin{prereg}
\begin{instruction}
Measurements which are not directly linked to the research question, but they are used in the analyses with primary factors to gain the precision of data interpretation. The addition of covariates should be justified by evidence. E.g. 
% * Relevant demographic features (e.g. age for child and adolescent population),
% * Order of condition (if the conditions are not counterbalanced)
\end{instruction}
\end{prereg}


\subsubsection{Other Measurements}
\begin{prereg}
\begin{instruction}
Measurements which are not directly linked to the research question, but they can provide information about the quality of the study. The instruments and the purpose of the measurements should be addressed. Such as:
% * Baseline demographic and clinical characteristics for each group [CONSORT 15]
% * Adverse events [CONSORT 19]
% * Expectancy (if blinded)
% * Measurements which monitor the compliance of the participants to restrictions (e.g. caffeine levels)
% * SOMETHING FOR LIGHT STUDIES?
\end{instruction}
\end{prereg}



\subsubsection{Covariates}
%% Covariates [CONSORT 12b “adjusted analysis”]
\begin{prereg}
\begin{instruction}
Measurements which are not directly linked to the research question, but they are used in the analyses with primary factors to gain the precision of data interpretation. The addition of covariates should be justified by evidence. E.g. 
% * Relevant demographic features (e.g. age for child and adolescent population),
% * Order of condition (if the conditions are not counterbalanced)
\end{instruction}
\end{prereg}


\subsubsection{Other Measurements}
\begin{prereg}
\begin{instruction}
Measurements which are not directly linked to the research question, but they can provide information about the quality of the study. The instruments and the purpose of the measurements should be addressed. Such as:
\begin{itemize}
    \item Baseline demographic and clinical characteristics for each group [CONSORT 15\%]
    \item Baseline demographic and clinical characteristics for each group [CONSORT 15\%]
    \item Adverse events [CONSORT 19; Expectancy (if blinded] 
    \item  Measurements which monitor the compliance of the participants to restrictions (e.g. caffeine levels; SOMETHING FOR LIGHT STUDIES?)
    \item Adverse events [CONSORT 19; expectancy (if blinded)]
\end{itemize}

\end{instruction}
\end{prereg}
