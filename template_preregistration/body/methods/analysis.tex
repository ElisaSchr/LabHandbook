\subsection{Statistical Analysis}

\subsubsection{Power Analysis}
%% [CONSORT #7a; CIE TN 011:2020 3.4]
\begin{prereg}
\begin{instruction}
Preliminary power analyses that is either based on data extracted from the literature or synthetic data. Discuss whether the sample size is justified using the power analysis, or given by fixed and limited resources.
\end{instruction}
\end{prereg}


\subsubsection{Pre-processing}
\begin{prereg}
\begin{instruction} 
This section should contain any details on processing, filtering, and other data transformations. This also includes any data exclusion and rejection rules (e.g., blinks in pupillometry data). This section should describe in unambiguous terms how raw data were processed. Code and algorithms used for preprocessing should be named and ideally made available.
\end{instruction}
\end{prereg}


\subsubsection{Confirmatory Analysis}
%% [CONSORT #12a; CIE TN 011:2020 3.4]
\begin{prereg}
\begin{instruction}
This section includes descriptions of any confirmatory analyses that are performed, how they are performed (i.e., which statistical test and which criterion used to accept a hypothesis), and how results from the statistical test would be interpreted. 
\end{instruction}
\end{prereg}

\begin{table}[h]
\footnotesize
\centering
\caption{Hypotheses and associated tests}
\label{tbl_hypotheses}

\begin{tabularx}{\textwidth}{XXXXXXXX}
\toprule
Confirmatory or exploratory &
  Hypothesis &
  Outcome measure &
  Sampling plan &
  Analysis plan &
  Interpretation \\ 
\midrule
\multirow{2}{*}{\textit{Confirmatory}} &
  H1 &
  Circadian phase shift (difference between DLMO) &
  \multirow{3}{*}{fixed N ($n=20$)} &
  Linear mixed model &
   \\
                     & H2 &  &  &  &  \\
\textit{Exploratory} &    &  &  &  &  \\ 
\bottomrule
\end{tabularx}

\end{table}


\subsubsection{Exploratory Analysis}
%% [CONSORT 18]
\begin{prereg}
\begin{instruction}
This section should include any ringfenced analyses, i.e. analyses that are exploratory but not confirmatory. 
\end{instruction}
\end{prereg}


\subsubsection{Data Storage and Privacy}
\begin{prereg}
\begin{instruction}
Describe how you anonymize the data (referring the participant with unrecognized numbers, using data sets for sensitive and non-sensitive information etc.), where to store, how long the data will be stored. In the case of promoting open science, describe when and how the data will be made available to the public.
\end{instruction}
\end{prereg}